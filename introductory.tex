%!TEX root=paper.tex

\begin{abstract}
\todo{Update abstract \& title}
  Python is one of the fastest growing programming languages of the moment. Flask, a Python-based web framework is the technology chosen by tens of thousands of web applications. Many of these Flask-based projects are small websites which can not afford to deploy an advanced service monitoring solution because they are too complicated to install. 

  In this paper, investigate how simple can it be to start monitoring a Flask API by developing and presenting \tool, a library that aims to provide insight into the utilization and performance of evolving services. We present the ease with which the library can be integrated in an already existing web application, discuss some of the visualization perspectives that the library provides and point to some future challenges for similar libraries.

\end{abstract}




\section{Introduction}

\todo{Intro to be rewritten; refer to VISSOFT paper as previous work, explain focus is on integration with version control/CI for evolution support, and use of unit tests as early performance indicator}

\todo{motivate with: monitoring in a world which is becoming more and more service-oriented is essential because it allows to perform three main types of actions: system adaptations to provide the request service at the desired level of quality, enabling flexibility in dealing with changing requirements and modes of operation, and enabling operational awareness through dashboards organizing the collected data~\cite{pernici2016monitoring}. In this work we focus on the last part and we discuss how a minimal effort probe-based solution for a particular type of web services can be used to facilitate the other two types of actions by involving the developer.}

\todo{explain: in~\cite{vogel2017low} we introduced the \tool an a high level with a focus on presenting its performance visualization aspects; in this work we only summarize these features by means of introducing some of the provided functionalities of the \tool. We focus on discussing how the tool is implemented and operating, provide a deeper look at its capabilities for integration with version control and continuous integration environments, introduce a mechanism for performance prediction, and provide an evaluation of the proposed approach based on a case study. }

%Every system is a distributed system nowadays \cite{cavage2013there}. Indeed a very large number of applications and web applications are nowadays implemented as two-tier architectures with a front-end implemented with web technologies and a service back-end.
%\ml{I'm not completely happy with this paragraph}
{\em There is no getting around it: you are building a distributed system} argues a recent article \cite{cavage2013there}. Indeed, even the simplest second-year student project is a web application implemented as two-tier architecture with a Javascript/HTML5 front-end a service backend, usually a REST API.

% \hfill mds
% Many contemporary programming languages are offering libraries, modules, or frameworks that facilitate the development of such architectures. 
Python is one of the most popular programming language choices for implementing the back-end of web applications. GitHub contains more than 500K open source Python projects and the Tiobe Index\footnote{TIOBE programming community index is a measure of popularity of programming languages, created and maintained by the TIOBE Company based in Eindhoven, the Netherlands} ranks Python as the 4th most popular programming language as of June 2016. An analysis of  StackOverflow from September 2017 argues that {\em ``Python has a solid claim to being the fastest-growing major programming language''}\footnote{https://stackoverflow.blog/2017/09/06/incredible-growth-python/}.
 
% possible flask summary
Within the Python community, Flask\footnote{\url{http://flask.pocoo.org/}} is a very popular web framework\footnote{More than 25K projects on GitHub (5\% of all Python projects) are implemented with Flask (cf. a GitHub search for ``language:Python Flask'')}. It provides simplicity and flexibility by implementing a bare-minimum web server, and thus advertises as a micro-framework. The Flask tutorial shows how setting up a simple Flask {\em ``Hello World''} web-service requires no more than 5 lines of Python code \cite{ flask:tutorial}.
% end of summary
 
Despite their popularity, to the best of our knowledge, there is no simple solution for monitoring the evolving performance of Flask web applications. Thus, every one of the developers of these projects faces one of the following options when confronted with the need of gathering insight into the runtime behavior of their implemented services: 

  \begin{enumerate}

    \item Use a commercial monitoring tool which treats the subject API as a black-box (e.g. Pingdom, Runscope). 
    % , Graphite+Graphana+statd etc.

    \item Implement their own ad-hoc analytics solution, having to reinvent basic visualization and interaction strategies. 

    \item Live without analytics insight into their services.

  \end{enumerate}

%\todo{For the first point in the list, we can also argue that analytics solutions like Google Analytics can be used, but they have no notion of versioning/integration with the development life cycle. Feel free to cite \cite{papazoglou2011managing} for service evolution purposes}

For projects on a budget (e.g. research, startups) the first and the second options are often not available due to time and financial constraints. Even when using 3rd-party analytics solutions, a critical insight into the evolution of the exposed services of the web application, is missing because such solutions have no notion of versioning and no integration with the development life cycle.~\cite{papazoglou2011managing}

To avoid projects ending up in the third situation, that of living without analytics, in this paper we present \tool~ --- a low-effort service monitoring library for Flask-based Python web services that is easy to integrate and enables the {\em agile assessment} of service evolution. \cite{Nier12b}

As a case study, on which we will illustrate our solution, we are going to use an open source API which, for several years, was in the third of the above-presented situations.

% In the next section, we will present a case study of an open source research API which was for a long time in the third situation presented above -- deployed without analytics insight.

\todo{Add signposting paragraph}



\section{Case Study: The API (Zeeguu)}
\label{sec:case}

%\todo{Either leave as is or add more info on usage by college if possible}

  \zee\footnote{\url{https://github.com/zeeguu-ecosystem/}} is a platform and an ecosystem of applications for accelerating vocabulary acquisition in a foreign language \cite{Lungu16}. 
%
  The architecture of the ecosystem has at its core an API and a series of satellite applications that together offer to a learner three main inter-dependent features:

  \begin{enumerate}

    \item Reader applications that provide effortless translations for those texts which are too difficult for the readers.

    \item Interactive exercises personally generated based on the preferences and past likes of the learner.

    \item Article recommendations which are personalized for the interests of the learner and come with difficulty estimation that helps the learner find articles with the appropriate difficulty.

  \end{enumerate}

  The core API implemented with Flask and Python provides correspondingly three types of functionality: contextual translations, article recommendations, and personalized exercise suggestions. In total the API provides a bit less than 50 endpoints, out of which probably a dozen are very frequently used. 
  % The development of the core API itself is a research project. 

  At the time of writing this article, the ecosystem consists of a reader web application, a web based exercises platform, and a smartwatch application, which are used at the moment of writing this article by more than \activeUserCount active users. The users come from a highschool, a language school, and some users are using it on their own, without any educational context. The highest load we observed until now on the API consisted of 12K requests in one day.

  \ml{add more numbers that show how impressive is our API}

  
  We will use this \zee API as a case study for this paper. 
  All the figures in this paper are captured from the actual deployment of \tool in the context of the \zee API. The figures are interactive offering basic data exploration capabilities: filter, zoom, and details on demand\cite{Shne99a}. The \tool deployment for the case study can be accessed publicly\footnote{\url{https://zeeguu.unibe.ch/api/dashboard}; username: {\em guest}, password: {\em soap-sac}}. 

 % \todo{create new account, update URL if necessary, update activeUserCount, add a couple of sentences of the current state of the case study with the college/language center}
% \ml{we should consider adding also one section in which the architecture/implementation and main features of the dashboard are presented before going on with discussing them in more depth in the following sections --- this should include a rundown on which views are provided from where (overview or per endpoint)}

\newpage



\section{Case Study: The Technology (Flask)}
\label{sec:tool}

 Flask is a microframework for Python. It IS USED BY THOUSANDS OF PROJECTS. ELABORATE MORE ON THIS... 
 \vspace{2cm}

 Flask depends on two external libraries: the Jinja2 template engine and the Werkzeug WSGI toolkit. 

 “Micro” does not mean that your whole web application has to fit into a single Python file (although it certainly can), nor does it mean that Flask is lacking in functionality. The “micro” in microframework means Flask aims to keep the core simple but extensible. Flask won’t make many decisions for you, such as what database to use. Those decisions that it does make, such as what templating engine to use, are easy to change. Everything else is up to you, so that Flask can be everything you need and nothing you don’t.

 Flask has many configuration values, with sensible defaults, and a few conventions when getting started. By convention, templates and static files are stored in subdirectories within the application’s Python source tree, with the names templates and static respectively. While this can be changed, you usually don’t have to, especially when getting started.

 A Python web application based on WSGI has to have one central callable object that implements the actual application. In Flask this is an instance of the Flask class. Each Flask application has to create an instance of this class itself and pass it the name of the module, but why can’t Flask do that itself?

\begin{lstlisting}[style=custompython]

from flask import Flask
app = Flask(__name__)

@app.route('/')
def index():
    return 'Hello World!'

\end{lstlisting}


\subsection*{ENSURE CLARITY: NOTHING PREVENTS GENERALIZATION}
In this article we take as an example the Flask case study
and of the \tool as a dashboard. 
But there is nothing that prevents the generalization of 
our approach to other languages and technologies. 


